
\documentclass[sigplan,review]{acmart}
\settopmatter{printfolios=true,printccs=false,printacmref=false}

\usepackage{color}
\usepackage{stmaryrd}  % double brackets
\usepackage{listings}
\lstset{
  mathescape=true
}
\usepackage{amssymb}

\renewcommand{\arraystretch}{1.2}

\renewcommand{\to}{\rightarrow}    % ->
\newcommand{\linto}{\multimap}     % -o
\newcommand{\grad}[1]{\nabla_S\lb #1 \rb}  % grad[#1]
\newcommand{\gradf}[1]{\nabla\! \mathit{#1}}  % Full Jacobian
\newcommand{\gradft}[1]{\Delta\mathit{#1}}  % Full Jacobian
\newcommand{\fwdDf}[1]{f'}  % Forward derivative, f'
\newcommand{\revDf}[1]{f`}  % Reverse derivative, f`
\newcommand{\lb}{\llbracket}
\newcommand{\rb}{\rrbracket}
\newcommand{\sel}[2]{\pi_{#1,#2}}
\newcommand{\iffun}{\mathit{if}}
\newcommand{\buildfun}{\mathit{build}}
\renewcommand{\vector}[1]{\mathit{Vec}\;#1}

\newcommand{\typ}[2]{#1 \! : \! #2}  % x:S, with less horizontal whitespace

\renewcommand{\dot}{.\,}               % dot with some space after
\newcommand{\real}{\mathbb{R}}       % R, the reals
\newcommand{\nat}{\mathbb{N}}        % N, the natural numbers
\newcommand{\darrow}{\Rightarrow}    % =>

% Linear maps
\newcommand{\lmapply}{\odot}   % Infix application\newcommand{\lmcomp}{\,\circ\,}   % Infix composition
\newcommand{\lmcomp}{\,\circ\,}   % Infix composition
\newcommand{\lmtrans}[1]{#1^{\top}}   % Trnaspose (takes an argument)
\newcommand{\lmpair}{\times}         % Infix pairing
\newcommand{\lmjoin}{\bowtie}        % Infix join
\newcommand{\lmzero}{\mathbf{0}}     % 0
\newcommand{\lmone}{\mathbf{1}}      % 1
\newcommand{\lmscalar}[1]{{\mathcal S}(#1)}      % S(k)
\newcommand{\lmlam}[1]{{\mathcal L}(#1)}      % L(k)
\newcommand{\lmlamt}[1]{{\mathcal M}(#1)}     % Transposed lambda
\newcommand{\lmbuild}{\mathcal B}             % Build linear map

\newcommand{\simon}[1]{{\color{red}SPJ: #1}}
\begin{document}

\title{Automatic differentation in Coconut}
\author{Tom Ellis}
\author{Simon Peyton Jones}
\author{Andrew Fitzgibbon}

\maketitle

% ------------------------------------
\section{Linear maps}

\begin{figure*}
\fbox{\begin{minipage}{\textwidth}
  $$
  \begin{array}{rr@{\hspace{2mm}}c@{\hspace{2mm}}ll}
                 & \multicolumn{3}{l}{\mbox{Operator \hspace{2em} Type}} & \mbox{Matrix interpretation} \\
\hline
    \mbox{Apply} & (\lmapply) & : & (s \linto t) \to (s \to t) \\
    \mbox{Compose} & (\lmcomp) & : & (s \linto t, r \linto s) \to (r \linto t)
           & \mbox{Matrix multiplication} \\
    \mbox{Pair}      & (\lmpair) & : & Field\; s \darrow (s \linto t_1, s \linto t_2) \to (s \linto (t_1,t_2))
           & \mbox{Vertical juxtaposition} \\
    \mbox{Join}  & (\lmjoin) & : & Field\; s \darrow (t_1 \linto s, t_2 \linto s) \to ((t_1,t_2) \linto s)
           & \mbox{Horizontal juxtaposition} \\
    \mbox{Sum}   & (+) & : & Field\; t \darrow (s \linto t) \to (s \linto t) \to (s \linto t) \\
    \mbox{Zero}  & \lmzero & : & Field\; t \darrow s \linto t & \mbox{Zero matrix}\\
    \mbox{Unit}  & \lmone  & : & s \linto s & \mbox{Identity matrix}\\
    \mbox{Scale} & \lmscalar{\cdot} & : & Field\; s \darrow s \to (s \linto s) \\
    \mbox{Transpose} & \lmtrans{\cdot} & : & (s \linto t) \to (t \linto s) & \mbox{Matrix transpose} \\
    \mbox{Lambda} & \lmlam{\cdot} & : & (\nat \to (s \linto t)) \to (s \linto (\nat \to t)) \\
    \mbox{TLambda} & \lmlamt{\cdot} & : & (\nat \to (t \linto s)) \to ((\nat \to t) \linto s) \\
    \mbox{Build}   & \lmbuild  & : & \nat \to (s \linto (\nat \to t)) \to (s \linto \vector{t}) \\
  \end{array}
  $$
\end{minipage}}
  \caption{Operations over linear maps} \label{fig:linear-maps}
\end{figure*}

\begin{figure}
  \fbox{\begin{minipage}{\columnwidth}
{\bf Semantics of linear maps}
  $$
  \begin{array}{rcl}
    (m_1 \lmcomp m_2) \lmapply x  & = & m_1 \lmapply (m_2 \lmapply x) \\
    (m_1 \lmpair m_2) \lmapply x  & = & (m_1 \lmapply x, m_2 \lmapply x) \\
    (m_1 \lmjoin m_2)  \lmapply (x_1,x_2) & = & (m_1 \lmapply x_1) + (m_2 \lmapply x_2) \\
    (m_1 + m_2)  \lmapply x & = & (m_1 \lmapply x) + (m_2 \lmapply x) \\
    \lmzero \lmapply x  & = & 0 \\
    \lmone \lmapply x & = & x \\
    \lmscalar{k} \lmapply x & = & k * x \\
    \lmlam{f} \lmapply x & = & \lambda n\dot (f n) \lmapply x \\
    \lmlamt{f} \lmapply g & = & \Sigma_i f(i) \lmapply g(i) \\
    \lmbuild(n,m) \lmapply x & = & build(n, m \lmapply x) \\
   \end{array}
  $$
\\[3mm]
  {\bf Rules for transposition of linear maps}
  $$
  \begin{array}{rcll}
    \lmtrans{(m_1 \lmcomp m_2)} & = & \lmtrans{m_2} \lmcomp \lmtrans{m_1} & \mbox{Note reversed order!}\\
    \lmtrans{(m_1 \lmpair m_2)} & = & \lmtrans{m_1} \lmjoin \lmtrans{m_2} \\
    \lmtrans{(m_1 \lmjoin m_2)} & = & \lmtrans{m_1} \lmpair \lmtrans{m_2} \\
    \lmtrans{(m_1 + m_2)} & = & \lmtrans{m_1} + \lmtrans{m_2} \\
    \lmtrans{\lmzero} & = & \lmzero \\
    \lmtrans{\lmone} & = & \lmone \\
    \lmtrans{\lmscalar{k}} & = & \lmscalar{k} \\
    \lmtrans{(\lmtrans{m})} & = & m \\
    \lmtrans{\lmlam{\lambda i. m}} & = & \lmlamt{\lambda i. \lmtrans{m}} \\
    \lmtrans{\lmlamt{\lambda i. m}} & = & \lmlam{\lambda i. \lmtrans{m}} \\
  \end{array}
  $$
\\[3mm]
  {\bf Laws for linear maps}
  $$
  \begin{array}{rcl}
    \lmzero \lmcomp m & = & \lmzero \\
    \lmone \lmcomp m & = & m \\
    m \lmcomp \lmzero & = & \lmzero \\
    m \lmcomp \lmone & = & m \\
    m \lmcomp (n_1 \lmjoin n_2) & = & (m \lmcomp n_1) \lmjoin (m \lmcomp n_2) \\
    (m_1 \lmjoin m_2) \lmcomp (n_1 \lmpair n_2) & = & (m_1 \lmcomp n_1) + (m_2 \lmcomp n_2) \\
    \lmscalar{k_1} \lmcomp \lmscalar{k_2} & = & \lmscalar{ k_1 * k_2 } \\
    \lmscalar{k_1} + \lmscalar{k_2} & = & \lmscalar{ k_1 + k_2 } \\
  \end{array}
  $$
    \end{minipage}
    }
    \caption{Laws for linear maps} \label{fig:lm-laws}
\end{figure}

A \emph{linear map}, $m : S \linto T$, is a function from $S$ to $T$,
satisfying these two properties:
$$
\begin{array}{rrcl}
  \forall \typ{x,y}{S} &  m \lmapply (x+y) & = & m \lmapply x + m \lmapply y \\
  \forall \typ{k}{\real}, \typ{x}{S} & k * (m \lmapply x) & = & m \lmapply (k * x)
\end{array}
$$
Here $(\lmapply) :: (s \linto t) \to (s \to t)$ is an operator that applies a linear map $(s \linto t)$
to an argument of type $s$.

\begin{itemize}
  \item Linear maps can be \emph{built} using the operators in (see Figure~\ref{fig:linear-maps}).
  \item The \emph{semantics} of a linear map is completely specified by saying
    what ordinary function it corresponds to; or, equivalently, by how it behaves when applied
    to an argument by $(\lmapply)$.  The semantics of each form of linear map are given in Figure~\ref{fig:lm-laws}
  \item Linear maps satisfy \emph{laws} given in Figure~\ref{fig:lm-laws}.
\end{itemize}

\subsection{Matrix interpretation of linear maps}

A linear map $m :: \real^m \linto \real^n$ is isomorphic to an matrix $\real^{n \times m}$ with $n$ rows and $m$ columns.

\subsection{Questions about linear maps}

\begin{itemize}
\item Do we need $\lmone$? After all $\lmscalar{1}$ does the same job.  But asking if $k=1$ is dodgy when $k$ is a float.
\item Do these laws fully define linear maps?
\item How do we transpose $\lmbuild$?
\end{itemize}
Notes
\begin{itemize}
\item In practice we allow n-ary versions of $m \lmjoin n$ and $m \lmpair n$.
\end{itemize}

% ------------------------------------
\section{The language}

\begin{figure}
  \fbox{\begin{minipage}{\columnwidth}
$$
      \begin{array}{rcll}
        f,g,h & ::= & \multicolumn{2}{l}{\mbox{Function}} \\
        x,y,z & ::= & \multicolumn{2}{l}{\mbox{Local variable (lambda-bound or let-bound)}} \\
        k & ::= & \multicolumn{2}{l}{\mbox{Literal constants}} \\
        \\
        \mathit{pgm} & ::= & \mathit{def}_1 \ldots \mathit{def}_n \\
        \mathit{def} & ::= & f(x) = e \\
        e & ::= & k & \mbox{Constant} \\
          & |   & x & \mbox{Local variable} \\
          & |   & f(e) & \mbox{Function call} \\
          & |   & (e_1,e_2) & \mbox{Pair} \\
          & |   & \lambda x \dot e & \mbox{Lambda} \\
          & |   & \mbox{\lstinline|let $x$ = $e_1$ in $\;e_2$|} \\
      \end{array}
 $$
\end{minipage}}
\caption{Syntax of the language} \label{fig:syntax}
\end{figure}
The syntax of our intermediate language is given in Figure~\ref{fig:syntax}.
Note that
\begin{itemize}
\item  Variables are divided into \emph{functions}, $f,g,h$; and \emph{local variables}, $x,y,z$,
  which are either function arguments or let-bound.
\item 
  The language is first order.  Functions are defined at top level; functions always appear in a call, never (say) as an argument to a vunction; in a call $f(e)$, the function $f$ is always a top-level-defined function, never a local variable.

\item Functions have exactly one argument. If you want more than one, pass a pair.

\item We allow ourselves to write functions infix where it is convenient.
  Thus $e_1 + e_2$ means the call $+(e1,e2)$, which applies the function $+$ to
  the pair $(e_1,e_2)$.  (So, like all other functions, $(+)$ has one argument.)
  Similarly the linear map $m_1 \lmpair m_2$ is short for $\lmpair(e_1,e_2)$.

\item Pairs are built-in, with selectors $\sel{1}{2}, \sel{2}{2}$.
  (In the real implementation, pairs are generalised to $n$-tuples.)

\item Conditionals are handled by a function $\iffun$.

\item Let-bindings are non-recursive. For now, at least, top-level
  functions are also non-recursive.  \simon{I think that top-level
    recursive functions might be OK, but I don't want to think about
    that yet.}

\item Lambda expressions are present, but without applications.
  If you can't apply a lambda, how can it be useful?  Answer: (only) as the
  argument to \lstinline|build| or \lstinline|fold|.
  \end{itemize}
\simon{But the rules for appliction in Figure~\ref{fig:lm-laws} involve applying those lambdas.  Ugh.}

\section{Automatic differentiation}

\begin{figure}
  \fbox{\begin{minipage}{\columnwidth}
$$
\begin{array}{ll}
  \mbox{\bf Original function}   & f : S \to T \\
  & f(x) = e \\[2mm]
  \mbox{\bf Full Jacobian}       & \gradf{f}  :  S \to (S \linto T) \\
  & \mbox{\lstinline|$\gradf{f}(x)$ = let $\;\gradf{x}$ = $\lmone\;$ in $\;\grad{e}$|} \\[2mm]
  \mbox{\bf Transposed Jacobian}   & \gradft{f}  :  S \to (T \linto S) \\
  & \mbox{\lstinline|$\gradft{f}(x)$ =  $\lmtrans{(\gradf{f}(x))}$|}  \\[2mm]
  \mbox{\bf Forward derivative}  & \fwdDf{f} : (S,S) \to T \\
  & \fwdDf{f}(x,dx) = \gradf{f}(x) \lmapply  dx \\[2mm]
  \mbox{\bf Reverse derivative}  & \revDf{f} : (S,T) \to S \\
  & \revDf{f}(x,dr) = \gradft{f}(x) \lmapply dr
\end{array}
$$
      {\bf Differentiation of an expression} \\
      \begin{center}
        If $e :: T$ then $\grad{e} :: S \linto T$
      \end{center}
$$
      \begin{array}{rcll}
        \grad{k} & = & \lmzero \\
        \grad{x} & = & \gradf{x} \\
        \grad{f(e)} & = & \gradf{f}(e) \lmcomp \grad{e} \\
        \grad{(e_1,e_2)} & = & \grad{e_1} \lmpair \grad{e_2} \\
        \grad{\buildfun(e_n,e)} & = & \lmbuild(e_n, \grad{e}) \\
        \grad{\lambda x \dot e} & = & \lmlam{\lambda x\dot \grad{e}} \\
        \grad{\mbox{\lstinline|let $\;x$ = $e_1\;$ in $\;e_2$|}}
        & = & \begin{array}[t]{l}
           \mbox{\lstinline|let $\;x\;$ = $\;e_1\;$ in|} \\
           \mbox{\lstinline|let $\;\gradf{x}\;$ = $\;\grad{e_1}\;$ in|} \\
           \mbox{\lstinline|$\grad{e_2}$|}
           \end{array}
      \end{array}
      $$
      \vspace{2mm}
      \bf{Built-in functions}
      $$
      \begin{array}{rcl}
        \gradf{+}      & :: & Field\; t \darrow (t,t) \to ((t,t) \linto t) \\
        \gradf{+}(x,y) & = & \lmone \lmjoin \lmone \\[1mm]
        \gradf{*}      & :: & Field\; t \darrow (t,t) \to ((t,t) \linto t) \\
        \gradf{*}(x,y) & = & \lmscalar{y} \lmjoin \lmscalar{x} \\[1mm]
        \gradf{\sel{1}{2}}      & :: & (t,t) \to ((t,t) \linto t) \\
        \gradf{\sel{1}{2}}(x) & = & \lmone \lmjoin \lmzero \\[1mm]
        \gradf{\iffun} & :: & (Bool,r,r) \to (((Bool,r,r) \linto r)) \\
        \gradf{\iffun}(True,t,f) & = & \lmzero \lmjoin \lmone \lmjoin \lmzero \\
        \gradf{\iffun}(False,t,f) & = & \lmzero \lmjoin \lmzero \lmjoin \lmone \\
        \ldots
        \end{array}
$$
\end{minipage}}
\caption{Automatic differentiation} \label{fig:ad}
\end{figure}

To perform source-to-source AD of a function $f$, we follow the plan
outlined in Figure~\ref{fig:ad}.  Specifically, starting with a
function definition \lstinline|f(x) = e|:

\begin{itemize}
\item Construct the full Jacobian $\gradf{f}$, and transposed full Jacobian $\gradft{f}$,
  using the tranformations in Figure~\ref{fig:ad}.
\item Optimise these two definitions, using the laws of linear maps
  in Figure~\ref{fig:lm-laws}.
\item Construct the forward derivative $\fwdDf{f}$ and reverse derivative $\revDf{f}$,
  as shown in Figure~\ref{fig:ad}.
\item Optimise these two definitions, to eliminate all linear maps. Specifically:
  \begin{itemize}
    \item Rather than \emph{calling} $\gradf{f}$ (in, say, $\fwdDf{f}$), instead \emph{inline} it.
    \item Similarly, for each local let-binding for a linear map, of form \lstinline|let $\;\gradf{x}$ = $e\;$ in $b$|,
      inline $\gradf{x}$ at each of its occurrences in $b$. This may duplicate $e$; but $\gradf{x}$ is a function
      that may be applied (via $\lmapply$) to many different arguments, and we want to specialise it for each
      such call.  (I think.)
    \item Optimise using the rules of $(\lmapply)$ in Figure~\ref{fig:lm-laws}.
    \item Use standard Common Subexpression Elimination (CSE)to recover any lost sharing.
  \end{itemize}
\end{itemize}

Note that
\begin{itemize}
\item The transformation is fully compositional; each function can be AD'd independently.
  For example, if a user-defined
  fuction $f$ calls another user-defined function $g$, we construct $\gradf{g}$ as
  described; and then construct $\gradf{f}$. The latter simply calls $\gradf{g}$.
  
\item We give the full Jacobian for some built-in functions in Figure~\ref{fig:ad}, including
  for conditionals ($\gradf{\iffun}$).

\item We may want to ANF-ise before AD to avoid gratuitous duplication.
  E.g.
$$
  \begin{array}{rcl}
    \multicolumn{3}{l}{\grad{sqrt(x+(y*z))}} \\
      & = & \gradf{sqrt}(x+(y*z)) \lmcomp \grad{x+(y*z)} \\
    & = & \gradf{sqrt}(x+(y*z)) \lmcomp  \gradf{+}(x, y*z) \\
     && \lmcomp (\grad{x} \lmpair \grad{y*z}) \\
    & = & \gradf{sqrt}(x+(y*z)) \lmcomp \gradf{+}(x, y*z) \\
    & & \lmcomp (\gradf{x} \lmpair (\gradf{*}(y,z) \lmcomp (\gradf{y} \lmpair \gradf{z}))) \\
  \end{array}
  $$
Note the duplication of $y*z$ in the result.
Of course, CSE may recover it.
\end{itemize}

\subsection{Build and lambda}

I'm not happy with the story for build and lambda. It started well:
\begin{itemize}
\item For a start, the treatment of lambda is very special: it works only
  for $(\nat \to t)$, which seems oddly assymetrical.

\item The AD for build (Fig~\ref{fig:ad}) looks sensible.
  But not great:  rather than use the generic case for a call $f(e)$,
  I used a special case for $\grad{\buildfun(e_n,e)}$; and I did
  an ad-hoc thing of not AD'ing the first argument to build, which
  seems very arbitrary.  The AD's version of build uses $\lmbuild$, whose
  signature is in Figure~\ref{fig:linear-maps}; and whose semantics is
  defined by how it behaves when applied (Figure~\ref{fig:lm-laws}).

\item The AD for lambda (Figure~\ref{fig:ad}) looks sensible.  It generates
  a new linear map $\lmlam$, whose
  signature is in Figure~\ref{fig:linear-maps}; and whose semantics is
  defined by how it behaves when applied (Figure~\ref{fig:lm-laws}).

\item But then things get trickier.  What is the transose of $\lmlam$?
  I invented $\lmlamt$ (Figure~\ref{fig:linear-maps}) as its transpose.
  There is a nice pattern here: $\lmpair$ and $\lmjoin$ are related
  in just the same way as $\lmlam$ and $\lmlamt$.

\item But I got stuck: what is the semantics of $\lmlamt$?
  There is a stab in Figure~\ref{fig:lm-laws}, but the $\Sigma_i$ is deeply
  suspicious, because it doesn't give the range of $i$.  That range comes
  from the enclosing build.
\end{itemize}

An alternative is to replace lambda with
$$
e ::= \ldots | \buildfun(n) i e
$$
where $i$ is a variable (of type $\nat$) that scopes over $e$.
So $\buildfun(n)~ i~ e$ is what we have been writing $\buildfun(n, \lambda i.e)$.
This looks simpler and more direct to me.

\subsection{Vectors}

Suppose we have $sum :: \vector{\real} \to \real$.  What is $\gradf{sum}$?
We may define its semantics, at least, by applying it:
$$
\begin{array}{rcl}
  \gradf{sum} & :: & \vector{\real} \to (\vector{\real} \linto \real) \\
  \gradf{sum}(v) \lmapply dv & = & sum(dv)
\end{array}
$$
That seem OK.  But what about product, which multiplies all the elements
of a vector together. If the vector had three elements we might have
$$
\begin{array}{l}
  \gradf{product}([x_1,x_2,x_3]) \lmapply [dx_1, dx_2, dx_3] \\
  \quad = (dx_1 * x_2 * x_3) + (dx_2 * x_1 * x_3) + (dx_3 * x_1 * x_2)
\end{array}
$$
This looks very unattractive as the number of elements grows.  Do we need
to use product?

This gives the clue that taking the derivative of $\mathit{fold}$ is
not going to be easy, maybe infeasible!  Much depends on the
particular lambda it appears.

\section{Implementation}

The implementation differs from this document as follows:
\begin{itemize}
\item Rather than pairs, the implementation supports $n$-ary tuples.
  Similary the linear maps $(\lmpair)$ and $\lmjoin$ are $n$-ary.
\item Functions definitions can take $n$ arguments, thus
  \begin{lstlisting}
   f(x,y,z) = e
  \end{lstlisting}
  This is treated as equivalent to
  \begin{lstlisting}
    f(t) = let x = $\sel{1}{3}$(t)
               y = $\sel{2}{3}$(t)
               z = $\sel{3}{3}$(t)
           in e
  \end{lstlisting}
\end{itemize}

\section{Questions}

\begin{itemize}
\item Do we need to add $\buildfun$ and array indexing?  Why can't we just work in terms of
  addition and multiplcation of vectors; and then use standard fusion to
  optimise the forward and reverse derivative?

\item We'll need sum/product of vectors.  What is $\gradf{sum}$ etc?

\end{itemize}  

\end{document}

do  C( s -o t) -> s -o C(t)
un             -> C(s) -o t

Given    m :: s -o (N -> t)
Wanted   mt :: (N -> t) -o s

L  :: (N -> (s -o t)) -> s -o (N -> t)
Lt ::                 -> (N -> s) -o t
