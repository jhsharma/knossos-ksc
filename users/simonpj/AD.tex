
\documentclass[sigplan,review]{acmart}
\settopmatter{printfolios=true,printccs=false,printacmref=false}

\usepackage{color}
\usepackage{stmaryrd}  % double brackets
\usepackage{listings}
\lstset{
  mathescape=true
}
\usepackage{amssymb}
\usepackage{enumitem}
\setlist[itemize,1]{itemsep=4pt}

\renewcommand{\arraystretch}{1.2}

\renewcommand{\to}{\rightarrow}    % ->
\newcommand{\linto}{\multimap}     % -o
\newcommand{\grad}[1]{\nabla_S\lb #1 \rb}  % grad[#1]
\newcommand{\gradf}[1]{\nabla\! \mathit{#1}}  % Full Jacobian
\newcommand{\gradft}[1]{\Delta\mathit{#1}}  % Full Jacobian
\newcommand{\fwdDf}[1]{f'}  % Forward derivative, f'
\newcommand{\revDf}[1]{f`}  % Reverse derivative, f`
\newcommand{\lb}{\llbracket}
\newcommand{\rb}{\rrbracket}
\newcommand{\sel}[2]{\pi_{#1,#2}}
\newcommand{\iffun}{\mathit{if}}
\newcommand{\buildfun}{\mathit{build}}
\newcommand{\sumfun}{\Sigma}
\newcommand{\sizefun}{size}
\newcommand{\deltafun}{\delta}
\newcommand{\indexfun}[2]{#1[#2]}   % Vector indexing
\renewcommand{\vector}[1]{\mathit{Vec}\;#1}

\newcommand{\typ}[2]{#1 \! : \! #2}  % x:S, with less horizontal whitespace

\renewcommand{\dot}{.\,}               % dot with some space after
\newcommand{\real}{\mathbb{R}}       % R, the reals
\newcommand{\nat}{\mathbb{N}}        % N, the natural numbers
\newcommand{\darrow}{\Rightarrow}    % =>

% Linear maps
\newcommand{\lmapply}{\odot}   % Infix application\newcommand{\lmcomp}{\,\circ\,}   % Infix composition
\newcommand{\lmcomp}{\,\circ\,}   % Infix composition
\newcommand{\lmtrans}[1]{#1^{\top}}   % Trnaspose (takes an argument)
\newcommand{\lmpair}{\times}         % Infix pairing
\newcommand{\lmjoin}{\bowtie}        % Infix join
\newcommand{\lmadd}{\oplus}        % Infix join
\newcommand{\lmzero}{\mathbf{0}}     % 0
\newcommand{\lmone}{\mathbf{1}}      % 1
\newcommand{\lmscalar}[1]{{\mathcal S}(#1)}      % S(k)
\newcommand{\lmlam}[1]{{\mathcal L}(#1)}      % L(k)
\newcommand{\lmlamt}[1]{{\mathcal M}(#1)}     % Transposed lambda
\newcommand{\lmbuild}{\mathcal B}             % Build linear map
\newcommand{\lmbuildt}{\mathcal C}             % Transposed Build linear map

\newcommand{\simon}[1]{{\color{red}SPJ: #1}}
\newcommand{\tom}[1]{{\color{red}TE: #1}}
\newcommand{\andrew}[1]{{\color{red}AWF: #1}}
\begin{document}

\title{Automatic differentation in Coconut}
\author{Tom Ellis}
\author{Simon Peyton Jones}
\author{Andrew Fitzgibbon}

\maketitle

% ------------------------------------
\section{Linear maps}

\begin{figure*}
\fbox{\begin{minipage}{\textwidth}
  $$
  \begin{array}{rr@{\hspace{2mm}}c@{\hspace{2mm}}ll}
      & \multicolumn{3}{l}{\mbox{Operator \hspace{2em} Type}} & \mbox{Matrix interpretation} \\
      & &&& \mbox{where $s = \real^m$, and $t = \real^n$} \\
\hline
    \mbox{Apply} & (\lmapply) & : & (s \linto t) \to (s \to t)
           & \mbox{Matrix/vector multiplication} \\
    \mbox{Compose} & (\lmcomp) & : & (s \linto t,\; r \linto s) \to (r \linto t)
           & \mbox{Matrix/matrix multiplication} \\
    \mbox{Sum}   & (\lmadd) & : & Field\; t \darrow (s \linto t,\; s \linto t) \to (s \linto t)
           & \mbox{Matrix addition } \\
    \mbox{Zero}  & \lmzero & : & Field\; t \darrow s \linto t & \mbox{Zero matrix}\\
    \mbox{Unit}  & \lmone  & : & s \linto s & \mbox{Identity matrix (square)}\\
    \mbox{Scale} & \lmscalar{\cdot} & : & Field\; s \darrow s \to (s \linto s) \\
    \mbox{Pair}      & (\lmpair) & : & Field\; s \darrow (s \linto t_1,\; s \linto t_2) \to (s \linto (t_1,t_2))
           & \mbox{Vertical juxtaposition} \\
    \mbox{Join}  & (\lmjoin) & : & Field\; s \darrow (t_1 \linto s,\; t_2 \linto s) \to ((t_1,t_2) \linto s)
           & \mbox{Horizontal juxtaposition} \\
    \mbox{Transpose} & \lmtrans{\cdot} & : & (s \linto t) \to (t \linto s) & \mbox{Matrix transpose} \\
%     \mbox{Lambda} & \lmlam{\cdot} & : & (\nat \to (s \linto t)) \to (s \linto (\nat \to t)) \\
%     \mbox{TLambda} & \lmlamt{\cdot} & : & (\nat \to (t \linto s)) \to ((\nat \to t) \linto s) \\
    \mbox{Build}   & \lmbuild  & : & (\nat,\; \nat \to (s \linto t)) \to (s \linto \vector{t}) \\
    \mbox{BuildT}   & \lmbuildt  & : & (\nat,\; \nat \to (t \linto s)) \to (\vector{t} \linto s) \\
  \end{array}
  $$
\end{minipage}}
  \caption{Operations over linear maps} \label{fig:linear-maps}
\end{figure*}

\begin{figure}
  \fbox{\begin{minipage}{\columnwidth}
{\bf Semantics of linear maps}
  $$
  \begin{array}{rcl}
    (m_1 \lmcomp m_2) \lmapply x  & = & m_1 \lmapply (m_2 \lmapply x) \\
    (m_1 \lmpair m_2) \lmapply x  & = & (m_1 \lmapply x, m_2 \lmapply x) \\
    (m_1 \lmjoin m_2)  \lmapply (x_1,x_2) & = & (m_1 \lmapply x_1) + (m_2 \lmapply x_2) \\
    (m_1 \lmadd m_2)  \lmapply x & = & (m_1 \lmapply x) + (m_2 \lmapply x) \\
    \lmzero \lmapply x  & = & 0 \\
    \lmone \lmapply x & = & x \\
    \lmscalar{k} \lmapply x & = & k * x \\
%    \lmlam{f} \lmapply x & = & \lambda n\dot (f n) \lmapply x \\
%    \lmlamt{f} \lmapply g & = & \Sigma_i f(i) \lmapply g(i) \\
    \lmbuild(n,f) \lmapply x & = & build(n, \lambda i. (f\; i) \lmapply x) \\
    \lmbuildt(n,f) \lmapply x & = & \sumfun\, \buildfun(n,\; \lambda i. (f\; i) \lmapply x[i] ) \\
   \end{array}
  $$
\\[3mm]
  {\bf Rules for transposition of linear maps}
  $$
  \begin{array}{rcll}
    \lmtrans{(m_1 \lmcomp m_2)} & = & \lmtrans{m_2} \lmcomp \lmtrans{m_1} & \mbox{Note reversed order!}\\
    \lmtrans{(m_1 \lmpair m_2)} & = & \lmtrans{m_1} \lmjoin \lmtrans{m_2} \\
    \lmtrans{(m_1 \lmjoin m_2)} & = & \lmtrans{m_1} \lmpair \lmtrans{m_2} \\
    \lmtrans{(m_1 \lmadd m_2)} & = & \lmtrans{m_1} \lmadd \lmtrans{m_2} \\
    \lmtrans{\lmzero} & = & \lmzero \\
    \lmtrans{\lmone} & = & \lmone \\
    \lmtrans{\lmscalar{k}} & = & \lmscalar{k} \\
    \lmtrans{(\lmtrans{m})} & = & m \\
    \lmtrans{\lmbuild( n,\; \lambda i. m )} & = & \lmbuildt( n,\; \lambda i. \lmtrans{m} ) \\
    \lmtrans{\lmbuildt( n,\; \lambda i. m )} & = & \lmbuild( n,\; \lambda i. \lmtrans{m} ) \\
%    \lmtrans{\lmlam{\lambda i. m}} & = & \lmlamt{\lambda i. \lmtrans{m}} \\
%    \lmtrans{\lmlamt{\lambda i. m}} & = & \lmlam{\lambda i. \lmtrans{m}} \\
  \end{array}
  $$
\\[3mm]
  {\bf Laws for linear maps}
  $$
  \begin{array}{rcl}
    \lmzero \lmcomp m & = & \lmzero \\
    m \lmcomp \lmzero & = & \lmzero \\
    \lmone \lmcomp m & = & m \\
    m \lmcomp \lmone & = & m \\
    m \lmadd \lmzero & = & m \\
    \lmzero \lmadd m & = & m \\
    m \lmcomp (n_1 \lmjoin n_2) & = & (m \lmcomp n_1) \lmjoin (m \lmcomp n_2) \\
    (m_1 \lmjoin m_2) \lmcomp (n_1 \lmpair n_2) & = & (m_1 \lmcomp n_1) \lmadd (m_2 \lmcomp n_2) \\
    \lmscalar{k_1} \lmcomp \lmscalar{k_2} & = & \lmscalar{ k_1 * k_2 } \\
    \lmscalar{k_1} \lmadd \lmscalar{k_2} & = & \lmscalar{ k_1 + k_2 } \\
  \end{array}
  $$
    \end{minipage}
    }
    \caption{Laws for linear maps} \label{fig:lm-laws}
\end{figure}

A \emph{linear map}, $m : S \linto T$, is a function from $S$ to $T$,
satisfying these two properties:
$$
\begin{array}{rrcl}
  \forall \typ{x,y}{S} &  m \lmapply (x+y) & = & m \lmapply x + m \lmapply y \\
  \forall \typ{k}{\real}, \typ{x}{S} & k * (m \lmapply x) & = & m \lmapply (k * x)
\end{array}
$$
Here $(\lmapply) :: (s \linto t) \to (s \to t)$ is an operator that applies a linear map $(s \linto t)$
to an argument of type $s$.

\begin{itemize}
  \item Linear maps can be \emph{built} using the operators in (see Figure~\ref{fig:linear-maps}).
  \item The \emph{semantics} of a linear map is completely specified by saying
    what ordinary function it corresponds to; or, equivalently, by how it behaves when applied
    to an argument by $(\lmapply)$.  The semantics of each form of linear map are given in Figure~\ref{fig:lm-laws}
  \item Linear maps satisfy \emph{laws} given in Figure~\ref{fig:lm-laws}.  Note that $(\lmcomp)$ and $\lmadd$ behave
    like multiplication and addition respectively.
\end{itemize}

\subsection{Matrix interpretation of linear maps}

A linear map $m :: \real^m \linto \real^n$ is isomorphic to an matrix $\real^{n \times m}$ with $n$ rows and $m$ columns.

\subsection{Questions about linear maps}

\begin{itemize}
\item Do we need $\lmone$? After all $\lmscalar{1}$ does the same job.  But asking if $k=1$ is dodgy when $k$ is a float.
\item Do these laws fully define linear maps?
\item How do we transpose $\lmbuild$?
\end{itemize}
Notes
\begin{itemize}
\item In practice we allow n-ary versions of $m \lmjoin n$ and $m \lmpair n$.
\end{itemize}

% ------------------------------------
\section{The language}

\begin{figure}
  \fbox{\begin{minipage}{\columnwidth}
$$
      \begin{array}{rcll}
        \multicolumn{4}{l}{\mbox{\bf Atoms}} \\
        f,g,h & ::= & \multicolumn{2}{l}{\mbox{Function}} \\
        x,y,z & ::= & \multicolumn{2}{l}{\mbox{Local variable (lambda-bound or let-bound)}} \\
        k & ::= & \multicolumn{2}{l}{\mbox{Literal constants}} \\
        \\
        \multicolumn{4}{l}{\mbox{\bf Terms}} \\
        \mathit{pgm} & ::= & \mathit{def}_1 \ldots \mathit{def}_n \\
        \mathit{def} & ::= & f(x) = e \\
        e & ::= & k & \mbox{Constant} \\
          & |   & x & \mbox{Local variable} \\
          & |   & f(e) & \mbox{Function call} \\
          & |   & (e_1,e_2) & \mbox{Pair} \\
          & |   & \lambda x \dot e & \mbox{Lambda} \\
          & |   & e_1 \; e_2 & \mbox{Application} \\
          & |   & \mbox{\lstinline|let $x$ = $e_1$ in $\;e_2$|} \\
        \\
        \multicolumn{4}{l}{\mbox{\bf Types}} \\
        \tau & ::= & \nat & \mbox{Natural numbers} \\
        & | & \real & \mbox{Real numbers} \\
        & | & (\tau_1, \tau_2) & \mbox{Pairs} \\
        & | & \vector{\tau} & \mbox{Vectors} \\
        & | & \tau_1 \to \tau_2 & \mbox{Functions} \\
        & | & \tau_1 \linto \tau_2 & \mbox{Linear maps} \\
      \end{array}
 $$
\end{minipage}}
\caption{Syntax of the language} \label{fig:syntax}
\end{figure}
The syntax of our intermediate language is given in Figure~\ref{fig:syntax}.
Note that
\begin{itemize}
\item  Variables are divided into \emph{functions}, $f,g,h$; and \emph{local variables}, $x,y,z$,
  which are either function arguments or let-bound.
\item 
  The language has a first order sub-language.  Functions are defined at top level;
  functions always appear in a call, never (say) as an argument to a
  vunction; in a call $f(e)$, the function $f$ is always a
  top-level-defined function, never a local variable.

\item Functions have exactly one argument. If you want more than one, pass a pair.

\item Pairs are built-in, with selectors $\sel{1}{2}, \sel{2}{2}$.
  (In the real implementation, pairs are generalised to $n$-tuples.)

\item Conditionals are handled by a function $\iffun$.

\item Let-bindings are non-recursive. For now, at least, top-level
  functions are also non-recursive.  \simon{I think that top-level
    recursive functions might be OK, but I don't want to think about
    that yet.}

\item Lambda expressions and applicatons are are present, so the language
  is higher order.  AD will only accept a subset of the language, in
  which lambdas appear only as an arguemnt to $\buildfun$.  But the
  \emph{output} of AD may include lambdas and application, as we shall see.
  \end{itemize}

\subsection{Built in functions}

\begin{figure}
  \fbox{\begin{minipage}{\columnwidth}
{\bf Built-in functions}
$$
\begin{array}{rcll}
  (+) & :: & Field \; t \darrow (t,t) \to t \\
  (*) & :: & Field \; t \darrow (t,t) \to t \\
  \sel{1}{2} & :: & (t_1,t_2) \to t_1 & \mbox{Selection} \\
  \sel{2}{2} & :: & (t_1,t_2) \to t_2 & \mbox{..ditto..} \\
  \deltafun & :: & Field \; t \darrow (\nat,\; \nat) \to t & \mbox{Delta-function} \\
  \buildfun & :: & (\nat,\; \nat \to t) \to \vector{t} & \mbox{Vector build} \\
  \indexfun{\cdot}{\cdot} & :: & (\vector{t},\; \nat) \to t & \mbox {Indexing} \\
  \sumfun & :: & Field \; t \darrow \vector{t} \to t & \mbox{Sum a vector} \\
  \iffun & :: & Bool \to r \to r \to r \\
\end{array}
$$
\\[3mm]
{\bf Derivatives of built-in functions}
      $$
      \begin{array}{rcl}
        \gradf{+}      & :: & Field\; t \darrow (t,t) \to ((t,t) \linto t) \\
        \gradf{+}(x,y) & = & \lmone \lmjoin \lmone \\[2mm]
        \gradf{*}      & :: & Field\; t \darrow (t,t) \to ((t,t) \linto t) \\
        \gradf{*}(x,y) & = & \lmscalar{y} \lmjoin \lmscalar{x} \\[2mm]
        \gradf{\sel{1}{2}}      & :: & (t,t) \to ((t,t) \linto t) \\
        \gradf{\sel{1}{2}}(x) & = & \lmone \lmjoin \lmzero \\[2mm]
        \gradf{\indexfun{\cdot}{\cdot}} & :: & (\vector{t},\; \nat) \to ((\vector{t},\; \nat) \linto t) \\
        \gradf{\indexfun{\cdot}{\cdot}}(v,i) & = & \lmbuildt( \sizefun(v), \lambda j. \deltafun(i,j)) \lmjoin \lmzero \\[2mm]
        \gradf{\sumfun} & :: & Field \; t \darrow \vector{t} \to (\vector{t} \linto t) \\
        \gradf{\sumfun}(v) & = & \lmbuildt( \sizefun(v),\; \lambda i. \lmone ) \\[2mm]
        \gradf{\iffun} & :: & (Bool,r,r) \to (((Bool,r,r) \linto r)) \\
        \gradf{\iffun}(True,t,f) & = & \lmzero \lmjoin \lmone \lmjoin \lmzero \\
        \gradf{\iffun}(False,t,f) & = & \lmzero \lmjoin \lmzero \lmjoin \lmone \\
        \ldots
        \end{array}
$$
\end{minipage}}
\caption{Built-in functions} \label{fig:built-in}
\end{figure}

The language has built-in functions shown in Figure~\ref{fig:built-in}.

We allow ourselves to write functions infix where it is convenient.
Thus $e_1 + e_2$ means the call $+(e1,e2)$, which applies the function $+$ to
the pair $(e_1,e_2)$.  (So, like all other functions, $(+)$ has one argument.)
Similarly the linear map $m_1 \lmpair m_2$ is short for $\lmpair(e_1,e_2)$.

We allow ourselves to write vector indexing using square brackets, thus $a[i]$.

Multiplication and addition are overloaded to work on any suitable type.
On vectors they work element-wise; if you want dot-product you have to program it.

The delta-function $\deltafun$ is defined thus:
$$
\begin{array}{rcll}
  \deltafun(i,j) & = & 1 & \mbox{if $i=j$} \\
  & = & 0 & \mbox{otherwise}
\end{array}
$$

\subsection{Vectors}

The language supposts one-dimensional vectors, of type $\vector{T}$,
whose elements have type $T$ (Figure~\ref{fig:syntax}).
It has built-in functions (Figure~\ref{fig:built-in}):
\begin{itemize}
\item $\buildfun :: (\nat,\; \nat \to t) \to \vector{t}$ for vector construction.
\item $\indexfun{\cdot}{\cdot} :: (\vector{t},\; \nat) \to t$ for indexing.
\item $\sumfun :: Field \; t \darrow \vector{t} \to t$ to add up the elements of a vector.
  \tom{I believe that for a vector $v$ of size $n$, $\sumfun v$ is the
    same as $\lmbuildt(n, const~ id)~ v$.  This may or may not be useful
    in reducing the size of the base language, should we want to do that.}
  \simon{I don't think so!  $\lmbuildt$ is a linear map, so you can't apply it to $v$.
    Maybe you mean $\lmbuildt(n, const~ id) \lmapply v$?  But that (Figure~\ref{fig:lm-laws}) is
    defined using $\sumfun$!}
\item $\sizefun :: \vector{t} \to \nat$ takes the size of a vector.
\item Arithmetic, $(*), (+)$ etc is overloaded to work over vectors, always elementwise.
\end{itemize}

% --------------------------------------
\section{Automatic differentiation}

\begin{figure}
  \fbox{\begin{minipage}{\columnwidth}
$$
\begin{array}{ll}
  \mbox{\bf Original function}   & f : S \to T \\
  & f(x) = e \\[2mm]
  \mbox{\bf Full Jacobian}       & \gradf{f}  :  S \to (S \linto T) \\
  & \mbox{\lstinline|$\gradf{f}(x)$ = let $\;\gradf{x}$ = $\lmone\;$ in $\;\grad{e}$|} \\[2mm]
  \mbox{\bf Transposed Jacobian}   & \gradft{f}  :  S \to (T \linto S) \\
  & \mbox{\lstinline|$\gradft{f}(x)$ =  $\lmtrans{(\gradf{f}(x))}$|}  \\[2mm]
  \mbox{\bf Forward derivative}  & \fwdDf{f} : (S,S) \to T \\
  & \fwdDf{f}(x,dx) = \gradf{f}(x) \lmapply  dx \\[2mm]
  \mbox{\bf Reverse derivative}  & \revDf{f} : (S,T) \to S \\
  & \revDf{f}(x,dr) = \gradft{f}(x) \lmapply dr
\end{array}
$$
      {\bf Differentiation of an expression} \\
      \begin{center}
        If $e :: T$ then $\grad{e} :: S \linto T$
      \end{center}
$$
      \begin{array}{rcll}
        \grad{k} & = & \lmzero \\
        \grad{x} & = & \gradf{x} \\
        \grad{f(e)} & = & \gradf{f}(e) \lmcomp \grad{e} \\
        \grad{(e_1,e_2)} & = & \grad{e_1} \lmpair \grad{e_2} \\
        \grad{\buildfun(e_n, \lambda i.e)} & = & \lmbuild(e_n, \lambda i. \grad{e}) \\
%         \grad{\lambda x \dot e} & = & \lmlam{\lambda x\dot \grad{e}} \\
        \grad{\mbox{\lstinline|let $\;x$ = $e_1\;$ in $\;e_2$|}}
        & = & \begin{array}[t]{l}
           \mbox{\lstinline|let $\;x\;$ = $\;e_1\;$ in|} \\
           \mbox{\lstinline|let $\;\gradf{x}\;$ = $\;\grad{e_1}\;$ in|} \\
           \mbox{\lstinline|$\grad{e_2}$|}
           \end{array}
      \end{array}
      $$
\end{minipage}}
\caption{Automatic differentiation} \label{fig:ad}
\end{figure}

To perform source-to-source AD of a function $f$, we follow the plan
outlined in Figure~\ref{fig:ad}.  Specifically, starting with a
function definition \lstinline|f(x) = e|:

\begin{itemize}
\item Construct the full Jacobian $\gradf{f}$, and transposed full Jacobian $\gradft{f}$,
  using the tranformations in Figure~\ref{fig:ad}.
\item Optimise these two definitions, using the laws of linear maps
  in Figure~\ref{fig:lm-laws}.
\item Construct the forward derivative $\fwdDf{f}$ and reverse derivative $\revDf{f}$,
  as shown in Figure~\ref{fig:ad}.
\item Optimise these two definitions, to eliminate all linear maps. Specifically:
  \begin{itemize}
    \item Rather than \emph{calling} $\gradf{f}$ (in, say, $\fwdDf{f}$), instead \emph{inline} it.
    \item Similarly, for each local let-binding for a linear map, of form \lstinline|let $\;\gradf{x}$ = $e\;$ in $b$|,
      inline $\gradf{x}$ at each of its occurrences in $b$. This may duplicate $e$; but $\gradf{x}$ is a function
      that may be applied (via $\lmapply$) to many different arguments, and we want to specialise it for each
      such call.  (I think.)
    \item Optimise using the rules of $(\lmapply)$ in Figure~\ref{fig:lm-laws}.
    \item Use standard Common Subexpression Elimination (CSE)to recover any lost sharing.
  \end{itemize}
\end{itemize}

Note that
\begin{itemize}
\item The transformation is fully compositional; each function can be AD'd independently.
  For example, if a user-defined
  fuction $f$ calls another user-defined function $g$, we construct $\gradf{g}$ as
  described; and then construct $\gradf{f}$. The latter simply calls $\gradf{g}$.

\item The AD transformation is \emph{partial}; that is, it does not work for every
  program. In particular, it fails when applield to a lambda, or an application; and,
  as we will see in Seciton~\ref{sec:vectors}, it requires that $\buildfun$ appears
  applied to a lambda.

\item We give the full Jacobian for some built-in functions in Figure~\ref{fig:ad}, including
  for conditionals ($\gradf{\iffun}$).
\end{itemize}

\subsection{Forward and reverse AD}

In \emph{``The essence of automatic differentiation''} Conal says (Section 12)
\begin{quote}
The AD algorithm derived in Section 4 and generalized in Figure 6 can be thought of as a family
of algorithms. For fully right-associated compositions, it becomes forward mode AD; for fully
left-associated compositions, reverse-mode AD; and for all other associations, various mixed modes.
\end{quote}
But the forward/reverse difference shows up quite differently here: it has nothing to do
with \emph{right-vs-left association}, and everything to do with \emph{transposition}.

This is mysterious.  Conal is not usually wrong.  I would like to
understand this better.
\tom{I was also puzzled by this.  Conal's claim is suspicious to me,
  but firstly it's very cool and secondly it's Conal, so I want it to
  be true and I still hope it is.}

\subsection{Avoiding duplication}

We may want to ANF-ise before AD to avoid gratuitous duplication.
  E.g.
$$
  \begin{array}{rcl}
    \multicolumn{3}{l}{\grad{sqrt(x+(y*z))}} \\
      & = & \gradf{sqrt}(x+(y*z)) \lmcomp \grad{x+(y*z)} \\
    & = & \gradf{sqrt}(x+(y*z)) \lmcomp  \gradf{+}(x, y*z) \\
     && \lmcomp (\grad{x} \lmpair \grad{y*z}) \\
    & = & \gradf{sqrt}(x+(y*z)) \lmcomp \gradf{+}(x, y*z) \\
    & & \lmcomp (\gradf{x} \lmpair (\gradf{*}(y,z) \lmcomp (\gradf{y} \lmpair \gradf{z}))) \\
  \end{array}
  $$
Note the duplication of $y*z$ in the result.
Of course, CSE may recover it.

\tom{Yes, although when I say ``AD'' I mean something that is distinct
  from what I mean by ``symbolic differentiation''.  In particular by
  ``AD'' I mean something that preserves sharing in a way that
  symbolic differentation doesn't.  Perhaps between us we should pin
  down some terminology.} \simon{I don't understand this. Perhaps you can make
  it precise?}

\tom{Consider $exp(exp(x))$.  I consider its ``symbolic derivative''
  to be $exp(exp(x)) exp(x)$ and its ``forward automatic derivative''
  to be $let~ y = exp(x)~ in~ exp(y) y$.  In other words, taking
  proper care of sharing is what makes AD AD and not just any old form
  of symbolic differentiation, in my personal nomenclature at least.
  Does that make it any clearer what I mean?}

\subsection{AD for vectors}

Like other built-in functions, each built-in function for vectors
has has its full Jacobian versions, defined in Figure~\ref{fig:built-in}.
You may enjoy checking that $\gradf{\sumfun}$ and
$\gradf{\indexfun{\cdot}{\cdot}}$ are correct!  I did think about having
a specialised linear map for indexing, rather than using $\lmbuildt$,
but then I needed its transposition, and that needed $\deltafun$ anyway.

But $\buildfun$ is an exception!  It is handled specially
by the AD transformation in Figure~\ref{fig:ad}; there is no $\gradf{\buildfun}$.
Moreover the AD transformation only works if the second argument of the build is
a lambda, thus $\buildfun(e_n, \lambda i.e)$.  I tried dealing with build and
lambdas separately, but failed (see Section~\ref{sec:build-lam-fail}).

\subsection{General folds}

We have $\sumfun :: \vector{\real} \to \real$.  What is $\gradf{\sumfun}$?
One way to define its semantics is by applying it:
$$
\begin{array}{rcl}
  \gradf{\sumfun} & :: & \vector{\real} \to (\vector{\real} \linto \real) \\
  \gradf{\sumfun}(v) \lmapply dv & = & \sumfun(dv)
\end{array}
$$
That is OK.  But what about product, which multiplies all the elements
of a vector together? If the vector had three elements we might have
$$
\begin{array}{l}
  \gradf{product}([x_1,x_2,x_3]) \lmapply [dx_1, dx_2, dx_3] \\
  \quad = (dx_1 * x_2 * x_3) + (dx_2 * x_1 * x_3) + (dx_3 * x_1 * x_2)
\end{array}
$$
This looks very unattractive as the number of elements grows.  Do we need
to use product?

This gives the clue that taking the derivative of $\mathit{fold}$ is
not going to be easy, maybe infeasible!  Much depends on the
particular lambda it appears.  So I have left out product, and made
no attempt to do general folds.


\section{Implementation}

The implementation differs from this document as follows:
\begin{itemize}
\item Rather than pairs, the implementation supports $n$-ary tuples.
  Similary the linear maps $(\lmpair)$ and $\lmjoin$ are $n$-ary.
\item Functions definitions can take $n$ arguments, thus
  \begin{lstlisting}
   f(x,y,z) = e
  \end{lstlisting}
  This is treated as equivalent to
  \begin{lstlisting}
    f(t) = let x = $\sel{1}{3}$(t)
               y = $\sel{2}{3}$(t)
               z = $\sel{3}{3}$(t)
           in e
  \end{lstlisting}
\end{itemize}

\section{Demo}

You can run the prototype by saying {\tt ghci Main}.

The function {\tt demo :: Def -> IO ()} runs the
prototype on the function provided as example.
Thus:
{\small
\begin{verbatim}
bash$ ghci Main

*Main> demo ex2

----------------------------
Original definition
----------------------------
fun f2(x)
  = let { y = x * x }
    let { z = x + y }
    y * z

----------------------------
Anf-ised original definition
----------------------------
fun f2(x)
  = let { y = x * x }
    let { z = x + y }
    y * z

----------------------------
The full Jacobian (unoptimised)
----------------------------
fun Df2(x)
  = let { Dx = lmOne() }
    let { y = x * x }
    let { Dy = lmCompose(D*(x, x), lmVCat(Dx, Dx)) }
    let { z = x + y }
    let { Dz = lmCompose(D+(x, y), lmVCat(Dx, Dy)) }
    lmCompose(D*(y, z), lmVCat(Dy, Dz))

----------------------------
The full Jacobian (optimised)
----------------------------
fun Df2(x)
  = let { y = x * x }
    lmScale( (x + y) * (x + x) + (x + y) * (x + x) )

----------------------------
Forward derivative (unoptimised)
----------------------------
fun f2'(x, dx)
  = lmApply(let { y = x * x }
            lmScale( (x + y) * (x + x) +
                     (x + y) * (x + x) ),
            dx)

----------------------------
Forward-mode derivative (optimised)
----------------------------
fun f2'(x, dx)
  = let { y = x * x }
    ((x + y) * (x + x) + (x + y) * (x + x)) * dx

----------------------------
Forward-mode derivative (CSE'd)
----------------------------
fun f2'(x, dx)
  = let { t1 = x + x * x }
    let { t2 = x + x }
    (t1 * t2 + t1 * t2) * dx

----------------------------
Transposed Jacobian
----------------------------
fun Rf2(x)
  = lmTranspose( let { y = x * x }
                 lmScale( (x + y) * (x + x) +
                         (x + y) * (x + x) ) )

----------------------------
Optimised transposed Jacobian
----------------------------
fun Rf2(x)
  = let { y = x * x }
    lmScale( (x + y) * (x + x) +
             (x + y) * (x + x) )

----------------------------
Reverse-mode derivative (unoptimised)
----------------------------
fun f2`(x, dr)
  = lmApply(let { y = x * x }
            lmScale( (x + y) * (x + x) +
                     (x + y) * (x + x) ),
            dr)

----------------------------
Reverse-mode derivative (optimised)
----------------------------
fun f2`(x, dr)
  = let { y = x * x }
    ((x + y) * (x + x) +
     (x + y) * (x + x)) * dr

----------------------------
Reverse-mode derivative (CSE'd)
----------------------------

fun f2`(x, dr)
  = let { t1 = x + x * x }
    let { t2 = x + x }
    (t1 * t2 + t1 * t2) * dr
\end{verbatim}
}


\section{Old stuff about Build and lambda} \label{sec:build-lam-fail}

I'm not happy with the story for build and lambda. It started well:
\begin{itemize}
\item For a start, the treatment of lambda is very special: it works only
  for $(\nat \to t)$, which seems oddly assymetrical.

\item The AD for build (Fig~\ref{fig:ad}) looks sensible.
  But not great:  rather than use the generic case for a call $f(e)$,
  I used a special case for $\grad{\buildfun(e_n,e)}$; and I did
  an ad-hoc thing of not AD'ing the first argument to build, which
  seems very arbitrary.  The AD's version of build uses $\lmbuild$, whose
  signature is in Figure~\ref{fig:linear-maps}; and whose semantics is
  defined by how it behaves when applied (Figure~\ref{fig:lm-laws}).

\item The AD for lambda (Figure~\ref{fig:ad}) looks sensible.  It generates
  a new linear map $\lmlam$, whose
  signature is in Figure~\ref{fig:linear-maps}; and whose semantics is
  defined by how it behaves when applied (Figure~\ref{fig:lm-laws}).

\item But then things get trickier.  What is the transose of $\lmlam$?
  I invented $\lmlamt$ (Figure~\ref{fig:linear-maps}) as its transpose.
  There is a nice pattern here: $\lmpair$ and $\lmjoin$ are related
  in just the same way as $\lmlam$ and $\lmlamt$.

\item But I got stuck: what is the semantics of $\lmlamt$?
  There is a stab in Figure~\ref{fig:lm-laws}, but the $\Sigma_i$ is deeply
  suspicious, because it doesn't give the range of $i$.  That range comes
  from the enclosing build.
\end{itemize}

An alternative is to replace lambda with
$$
e ::= \ldots | \buildfun(n) i e
$$
where $i$ is a variable (of type $\nat$) that scopes over $e$.
So $\buildfun(n)~ i~ e$ is what we have been writing $\buildfun(n, \lambda i.e)$.
This looks simpler and more direct to me.
\tom{Is there something to stop me writing $let f = \lambda i.e~ in~
  \buildfun(n)~ i~ f(i)$ in an attempt to resurrect original behaviour?}
\simon{Yes: $f(i)$ requires $f$ to be top level -- see Section 2.  But
  the solution I adopted was not to have a new language construct, but
  rather to differentiate only $\buildfun(n, \lambda i.e)$.  Any other
  use of $\buildfun$ will fail.}
\end{document}

do  C( s -o t) -> s -o C(t)
un             -> C(s) -o t

Given    m :: s -o (N -> t)
Wanted   mt :: (N -> t) -o s

L  :: (N -> (s -o t)) -> s -o (N -> t)
Lt ::                 -> (N -> s) -o t
