
\documentclass[sigplan,review]{acmart}
\settopmatter{printfolios=true,printccs=false,printacmref=false}

\usepackage{color}
\usepackage{stmaryrd}  % double brackets
\usepackage{listings}
\lstset{
  mathescape=true
}
\usepackage{amssymb}

\renewcommand{\arraystretch}{1.2}

\renewcommand{\to}{\rightarrow}    % ->
\newcommand{\linto}{\multimap}     % -o
\newcommand{\grad}[1]{\nabla\lb \, #1 \, \rb}  % grad[#1]
\newcommand{\gradf}[1]{\nabla\! #1}  % Full Jacobian
\newcommand{\fwdDf}[1]{f'}  % Forward derivative, f'
\newcommand{\revDf}[1]{f`}  % Reverse derivative, f`
\newcommand{\lb}{\llbracket}
\newcommand{\rb}{\rrbracket}


\newcommand{\typ}[2]{#1 \! : \! #2}  % x:S, with less horizontal whitespace

\renewcommand{\dot}{.\,}               % dot with some space after
\newcommand{\real}{\mathbb{R}}       % R, the reals
\newcommand{\nat}{\mathbb{N}}        % N, the natural numbers
\newcommand{\darrow}{\Rightarrow}    % =>

% Linear maps
\newcommand{\lmapply}{\odot}   % Infix application\newcommand{\lmcomp}{\,\circ\,}   % Infix composition
\newcommand{\lmcomp}{\,\circ\,}   % Infix composition
\newcommand{\lmtrans}[1]{#1^{\top}}   % Trnaspose (takes an argument)
\newcommand{\lmpair}{\times}         % Infix pairing
\newcommand{\lmjoin}{\bowtie}        % Infix join
\newcommand{\lmzero}{\mathbf{0}}     % 0
\newcommand{\lmone}{\mathbf{1}}      % 1
\newcommand{\lmscalar}[1]{{\mathcal S}(#1)}      % S(k)
\newcommand{\lmlam}[1]{{\mathcal L}(#1)}      % L(k)

\newcommand{\simon}[1]{{\color{red}SPJ: #1}}
\begin{document}

\title{Autommatic differentation in Coconut}
\author{Tom Ellis}
\author{Simon Peyton Jones}
\author{Andrew Fitzbiggon}

\maketitle

% ------------------------------------
\section{Linear maps}

\begin{figure*}
\fbox{\begin{minipage}{\textwidth}
  $$
  \begin{array}{rr@{\hspace{2mm}}c@{\hspace{2mm}}ll}
                 & \multicolumn{3}{l}{\mbox{Operator \hspace{2em} Type}} & \mbox{Matrix interpretation} \\
\hline
    \mbox{Apply} & (\lmapply) & : & (s \linto t) \to (s \to t) \\
    \mbox{Compose} & (\lmcomp) & : & (s \linto t) \to (r \linto s) \to (r \linto t) \\
           & \mbox{Matrix multiplication} \\
    \mbox{Pair}      & (\lmpair) & : & Field\; s \darrow (s \linto t_1) \to (s \linto t_2) \to (s \linto (t_1,t_2))
           & \mbox{Vertical juxtaposition} \\
    \mbox{Join}  & (\lmjoin) & : & Field\; t \darrow (s_1 \linto t) \to (s_2 \linto t) \to ((s_1,s_2) \linto t)
           & \mbox{Horizontal juxtaposition} \\
    \mbox{Sum}   & (+) & : & Field\; t \darrow (s \linto t) \to (s \linto t) \to (s \linto t) \\
    \mbox{Zero}  & \lmzero & : & Field\; t \darrow s \linto t & \mbox{Zero matrix}\\
    \mbox{Unit}  & \lmone  & : & s \linto s & \mbox{Identity matrix}\\
    \mbox{Scalar} & \lmscalar{\cdot} & : & Field\; s \darrow s \to (s \linto s) \\
    \mbox{Lambda} & \lmlam{\cdot} & : & (\nat \to (s \linto t)) \to (s \linto (\nat \to t)) \\
    \mbox{Transpose} & \lmtrans{\cdot} & : & (s \linto t) \to (t \linto s) & \mbox{Matrix transpose} \\
  \end{array}
  $$
\end{minipage}}
  \caption{Operations over linear maps} \label{fig:linear-maps}
\end{figure*}

\begin{figure}
  \fbox{\begin{minipage}{\columnwidth}
{\bf Rules for application of linear maps}
  $$
  \begin{array}{rcl}
    (m_1 \lmcomp m_2) \lmapply x  & = & m_1 \lmapply (m_2 \lmapply x) \\
    (m_1 \lmpair m_2) \lmapply x  & = & (m_1 \lmapply x, m_2 \lmapply x) \\
    (m_1 \lmjoin m_2)  \lmapply (x_1,x_2) & = & (m \lmapply x_1) + (m \lmapply x_2) \\
    (m_1 + m_2)  \lmapply x & = & (m_1 \lmapply x) + (m_2 \lmapply x) \\
    \lmzero \lmapply x  & = & 0 \\
    \lmone \lmapply x & = & x \\
    \lmscalar{k} \lmapply x & = & k * x \\
    \lmlam{f} \lmapply x & = & \lambda n\dot (f n) \lmapply x \\
   \end{array}    
  $$    
\\[3mm]
  {\bf Rules for transposition of linear maps}
  $$
  \begin{array}{rcl}
    \lmtrans{(m_1 \lmcomp m_2)} & = & \lmtrans{m_2} \lmcomp \lmtrans{m_1} \\
    \lmtrans{(m_1 \lmpair m_2)} & = & \lmtrans{m_1} \lmjoin \lmtrans{m_2} \\
    \lmtrans{(m_1 \lmjoin m_2)} & = & \lmtrans{m_1} \lmpair \lmtrans{m_2} \\
    \lmtrans{(m_1 + m_2)} & = & \lmtrans{m_1} + \lmtrans{m_2} \\
    \lmtrans{\lmzero} & = & \lmzero \\
    \lmtrans{\lmone} & = & \lmone \\
    \lmtrans{\lmscalar{k}} & = & \lmscalar{k} \\
    \lmtrans{(\lmtrans{m})} & = & m \\
    \lmtrans{\lmlam{f}} & = & ??? \\
  \end{array}
  $$
\\[3mm]
  {\bf Laws for linear maps}
  $$
  \begin{array}{rcl}
    \lmzero \lmcomp m & = & \lmzero \\
    \lmone \lmcomp m & = & m \\
    m \lmcomp \lmzero & = & \lmzero \\
    m \lmcomp \lmone & = & m \\
    m \lmcomp (n_1 \lmjoin n_2) & = & (m \lmcomp n_1) \lmjoin (m \lmcomp n_2) \\
    (m_1 \lmjoin m_2) \lmcomp (n_1 \lmpair n_2) & = & (m_1 \lmcomp n_1) + (m_2 \lmcomp n_2) \\
    \lmscalar{k_1} \lmcomp \lmscalar{k_2} & = & \lmscalar{ k_1 * k_2 } \\
    \lmscalar{k_1} + \lmscalar{k_2} & = & \lmscalar{ k_1 + k_2 } \\
  \end{array}
  $$
    \end{minipage}
    }
    \caption{Laws for linear maps} \label{fig:lm-laws}
\end{figure}

A \emph{linear map}, $m : S \linto T$, is a function from $S$ to $T$,
which can be applied to an argument of type $S$ with the $(\lmapply)$ operator
(see Figure~\ref{fig:linear-maps}).

Any function with these two properties is a linear map:
$$
\begin{array}{rrcl}
  \forall \typ{x,y}{S} &  m \lmapply ((x+y) & = & m \lmapply x + m \lmapply y \\
  \forall \typ{k}{\real}, \typ{x}{S} & k * (m \lmapply x) & = & m \lmapply (k * x)
\end{array}
$$
Questions
\begin{itemize}
\item Do we need $\lmone$; after all $\lmscalar{1}$ does the same job.  But asking if $k=1$ is dodgy when $k$ is a float.
\item Do these fully define linear maps?
\end{itemize}
Notes
\begin{itemize}
\item In practice we allow n-ary versions of $m \lmjoin n$ and $m \lmpair n$.
\end{itemize}

% ------------------------------------
\section{The language}

\begin{figure}
  \fbox{\begin{minipage}{\columnwidth}
$$
      \begin{array}{rcll}
        f,g,h & ::= & \multicolumn{2}{l}{\mbox{Function}} \\
        x,y,z & ::= & \multicolumn{2}{l}{\mbox{Local variable (lambda-bound or let-bound)}} \\
        k & ::= & \multicolumn{2}{l}{\mbox{Literal constants}} \\
        \\
        \mathit{pgm} & ::= & \mathit{def}_1 \ldots \mathit{def}_n \\
        \mathit{def} & ::= & f(x) = e \\
        e & ::= & k & \mbox{Constant} \\
          & |   & x & \mbox{Local variable} \\
          & |   & f(e) & \mbox{Function call} \\
        & |   & (e_1,e_2) & \mbox{Pair} \\
        & |   & \lambda x \dot e & \mbox{Lambda} \\
        & |   & \lstinline|let $x$ = $e_1$ in $\;e_2$| \\
      \end{array}
 $$
\end{minipage}}
\caption{Syntax of the language} \label{fig:syntax}
\end{figure}
The syntax of our intermediate language is given in Figure~\ref{fig:syntax}.

\begin{itemize}
  \item \simon{It's a bit odd to put in lambda without application;
  but our lambdas are pretty restircted to just $\nat \to T$ functions.}
  \item In practice we allow n-ary tuples, not just 2-tuples.
\end{itemize}


\section{Automatic differentiation}

\begin{figure}
  \fbox{\begin{minipage}{\columnwidth}
\bf{Dfferentiation of an expression} \\
$$
      \begin{array}{rcll}
        \grad{k} & = & \lmzero \\
        \grad{x} & = & \gradf{x} \\
        \grad{f(e)} & = & \gradf{f}(e) \lmcomp \grad{e} \\
        \grad{(e_1,e_2)} & = & \grad{e_1} \lmpair \grad{e_2} \\
        \grad{\lambda x \dot e} & = & \lmlam{\lambda x\dot \grad{e}} \\
        \grad{\lstinline|let $x$ = $e_1$ in $\;e_2$|}
        & = & \begin{array}[t]{l}
           \lstinline|let $x\;$ = $\;e_1\;$ in| \\
           \lstinline|let $\gradf{x}\;$ = $\;\grad{e_1}\;$ in| \\
           \lstinline|$\grad{e_2}$|
           \end{array}
      \end{array}
      $$
      \vspace{2mm}
      \bf{Built-in functions}
      $$
      \begin{array}{rcl}
        \gradf{+}(x,y) & = & \lmone \lmjoin \lmone \\
        \gradf{*}(x,y) & = & \lmscalar{y} \lmjoin \lmscalar{x} \\
        \ldots
        \end{array}
      $$
\end{minipage}}
\caption{Automatic differentiation} \label{fig:ad}
\end{figure}

Automatic differentiation is described by Figure~\ref{fig:ad}.
$$
\begin{array}{ll}
  \mbox{Original function}   & f : S \to T \\
  & f(x) = e \\[2mm]
  \mbox{Full Jacobian}       & \gradf{f}  :  S \to (S \linto T) \\
  & \lstinline|$\gradf{f}(x)$ = let $\gradf{x}$ = $\lmone$ in $\grad{e}$| \\[2mm]
  \mbox{Forward derivative}  & \fwdDf{f} : S \to S \to T \\
  & \fwdDf{f}(x,dx) = \gradf{f} \lmapply  dx \\[2mm]
  \mbox{Reverse derivative}  & \revDf{f} : S \to T \to S \\
  & \revDf{f}(x,dr) = \lmtrans{(\gradf{f}(x))} \lmapply dr
\end{array}
$$
Notes
\begin{itemize}
\item We must ANF-ise the fucntion before AD to avoid gratuitous duplication.
  E.g.
$$
  \begin{array}{rcl}
    \multicolumn{3}{l}{\grad{sqrt(x+(y*z))}} \\
      & = & \gradf{sqrt}(x+(y*z)) \lmcomp \grad{x+(y*z)} \\
      & = & \gradf{sqrt}(x+(y*z)) \lmcomp  \gradf{+}(x, y*z) \lmcomp (\grad{x} \lmpair \grad{y*z}) \\
    & = & \gradf{sqrt}(x+(y*z)) \lmcomp \gradf{+}(x, y*z) \\
    & & \lmcomp (\gradf{x} \lmpair (\gradf{*}(y,z) \lmcomp (\gradf{y} \lmpair \gradf{z}))) \\
  \end{array}
  $$
  Note the duplication of $y*z$ in the result.
\end{itemize}
\end{document}

do  C( s -o t) -> s -o C(t)
un             -> C(s) -o t

Given    m :: s -o (N -> t)
Wanted   mt :: (N -> t) -o s

L  :: (N -> (s -o t)) -> s -o (N -> t)
Lt ::                 -> (N -> s) -o t
